\section{Results}
\subsection{Hypothese 3}

Aus den Auswertungen der Plenarprotokolle sind folgende Ergebnisse entstanden:
\begin{center}
% latex table generated in R 3.5.0 by xtable 1.8-3 package
% Sun Nov 18 13:22:12 2018
\begin{table}[ht]
	\centering
	\begin{tabular}{lrrrrrrrl}
		\hline
		Partei & Beifall & Heiterkeit & Kommentar & Lachen & Widerspruch & Zuruf & Summe & Periode \\ 
		\hline
		\hline
		SPD & 12311 & 318 & 3346 & 231 & 109 & 373 & 16315 & 17-18 \\ 
		CDU/CSU & 11659 & 333 & 3847 & 112 &  53 & 223 & 16004 & 17-18 \\ 
		DIE LINKE & 9462 & 125 & 2927 & 142 & 107 & 367 & 12763 & 17-18 \\ 
		GRÜNE & 7417 & 131 & 4205 &  95 &  69 & 335 & 11917 & 17-18 \\ 
		FDP & 8470 & 258 & 2782 &  86 &  36 & 249 & 11632 & 17-18 \\ 
		AfD & 7074 & 129 & 3557 & 291 & 114 & 734 & 11165 & 17-18 \\ 
		\hline
		CDU/CSU & 32638 & 856 & 10018 & 234 & 150 & 1047 & 43812 & 13-17 \\ 
		SPD & 32217 & 934 & 6087 & 124 & 159 & 615 & 39407 & 13-17 \\ 
		GRÜNE & 14148 & 158 & 15363 & 116 & 131 & 791 & 29800 & 13-17 \\ 
		DIE LINKE & 20789 & 282 & 8218 & 153 & 210 & 1221 & 29451 & 13-17 \\ 
		\hline
		SPD & 27330 & 731 & 24078 & 863 & 496 & 2264 & 53064 & 09-13 \\ 
		CDU/CSU & 34131 & 633 & 13515 & 326 & 272 & 1584 & 48719 & 09-13 \\ 
		FDP & 32808 & 577 & 10484 & 333 & 280 & 1499 & 44323 & 09-13 \\ 
		DIE LINKE & 19275 & 308 & 7583 & 299 & 268 & 1199 & 27503 & 09-13 \\ 
		GRÜNE & 11716 & 199 & 13391 & 213 & 178 & 634 & 25537 & 09-13 \\ 
		\hline
	\end{tabular}
\end{table}
\end{center}

\subsubsection{Klatschen für andere Fraktionen}

 Beifall ist eine sehr deutliche Form der Zustimmung, die einfach zu erkennen und leicht interpretierbar ist. Klatscht eine Partei für eine andere, so lässt sich daraus schließen, dass die Partei der Rede zustimmen. Aus den Bundestagsprotokollen geht dabei hervor, welche Partei wann klatscht und ob die ganze Partei in den Beifall einstimmt, oder nur einzelne Abgeordnete. Unsere Analyse hat aus den Protokolle der vergangenen zwei Perioden ausgewertet, welche Partei für welche klatscht und damit Zustimmung signalisiert.   \\

\includegraphics[width=\linewidth]{Grafiken/13_17WerfürWen.png}\\
\includegraphics[width=\linewidth]{Grafiken/13_17WerfürWen_prozent.png}\\
Es zeigt sich, dass die Koalitionsparteien am über alle drei Legislaturperioden hinweg am meisten für einander applaudieren. Sowohl 2009-2013 die FDP, die sehr oft für die CDU/CSU klatscht, als auch die SPD und die CDU/CSU, die 2013-2018 oft für einander applaudieren. \\
....
\subsubsection{Klatschen für die eigene Fraktion}
 \noindent Je häufiger eine Partei für sich selbst klatscht im Verhältnis zum Klatschen für andere Parteien, desto eher ist sie in einer isolierten Position. Sie stimmt nur den eigenen Inhalten zu und hat kaum eine inhaltliche Position mit einer anderen Partei gemeinsam. Eine Analyse über den "Eigenklatschanteil" kann also Aufschluss darüber geben, wie Isoliert eine Partei ist. \\

\includegraphics[width=\linewidth]{Grafiken/17_eigenklatschanteil.png}

Die AfD ist demnach die isolierteste Partei. Betrachtet man den "Eigenklatschanteil" im Zeitverlauf zeigt sich, dass auch andere Parteien sich in der momentanen Legislaturperiode zunehmend zu isolieren scheinen. Dass die Kurve am Ende, also gegen Oktober, wieder abfällt, könnte darauf zurückzuführen sein, dass im Monat Oktober zu Beginn unserer Analyse, noch nicht alle Bundestagsreden veröffentlicht waren und daher im Oktober Messzeitpunkte fehlen. \\
  
\includegraphics[width=\linewidth]{Grafiken/13_17Zeitverlauf_Klatschen.png}\\

\subsubsection{Geschlossenheit der Parteien}
Ein geschlossenes Auftreten von Parteien nur anhand von gemeinsamen Interaktionen fest zu machen, ist nur begrenzt aussagekräftig. Dennoch sind neben einem kohärenten Inhalt gemeinsame Aktionen ein deutliches Zeichen nach außen, ob eine Partei geschlossen auftritt, oder nicht. Gemeinsamer Beifall ist demnach ein Indikator für Geschlossenheit. \\

\includegraphics[width=\linewidth]{Grafiken/Geschlossenheit13.png}
\includegraphics[width=\linewidth]{Grafiken/Geschlossenheit17.png}\\

Es zeigt sich, dass weder die Hypothese bestätigt werden kann, dass sich eine große Veränderung bezüglich der Geschlossenheit gegeben hat, noch, dass die AfD als Newcomer-Partei weder besonders geschlossen, noch besonders gespalten auftritt. Interessanter ist eher, dass sich der Fraktionsstreit der CDU mit der CSU deutlich in dem Ergebniss zeigt, dass die CDU/CSU plötzlich auf dem letzten Platz ist und in fast der Hälfte der Fälle nur einzelne Abgeordnete Klatschen und nicht die ganze Partei. 

\subsubsection{Isolierung in der Wortwahl} 
Die Grafik spricht für sich. Die einzige Partei, die "Deutschland und deutsch" häufiger verwendet, als "Mensch" ist die AfD. \\

\includegraphics[width=\linewidth]{Grafiken/17_HäufigsteWörter.png}\\

\subsection{Negative Interaktionen}

"Lachen" wird vom stenografischen Dienst in Abgrenzung zur "Heiterkeit" als negative, Aktion beschrieben. Ebenso wird "Widerspruch" und "Zuruf", in Abgrenzung zu einem Kommentar, der sowohl positiv, als auch negativ sein kann, als negative Aktion beschrieben. 

%\includegraphics[width=\linewidth]{Grafiken/WelchePArteiLachtwieviel.png}

Unsere Vermutung, dass die AfD am häufigsten Lacht ist damit bestätigt. 

\includegraphics[width=\linewidth]{Grafiken/13_17negativ_perc_bunt.png}\\

%\includegraphics[width=\linewidth]{Grafiken/13_17negativ_leg.png}\\
%\includegraphics[width=\linewidth]{Grafiken/13_eigenklatschanteil.png}







\subsection {Readability Index}
Der Flesch-Reading Index, der auf englische Texte optimiert
ist wurde von Amstad (\cite{amstad_wie_1978}) auf deutsprachige Texte angepasst. Die
Werte wurden mit der Funktion \texttt{flesch()} mit dem Parameter {\verb de } aus dem R-Package KoRpus (\cite{michalke_korpus:_2018}) berechnet.\\

\fbox{\parbox{\linewidth}{\[r_{German} = 80 - 58.5 * \frac{x}{y} - \frac{w}{s} \] \textbf{w}: Gesamtanzahl von Wörtern \textbf{y}: Gesamtanzahl von Silben \textbf{s}: Gesamtanzahl von Sätzen}}\\


\noindent Dafür wurden die Texte zunächst in Buchstaben aufgeteilt. Dies wurde mit
der \texttt{tokenize()} funktion aus dem ``koRpus'' Package gemacht.

\begin{longtable}[]{@{}lll@{}}
	\toprule
	Flesch-Reading-Ease-Score Von \ldots{} bis unter \ldots{} & Lesbarkeit &
	Verständlich für\tabularnewline
	\midrule
	\endhead
	0--30 & Sehr schwer & Akademiker\tabularnewline
	30--50 & Schwer &\tabularnewline
	50--60 & Mittelschwer &\tabularnewline
	60--70 & Mittel & 13--15-jährige Schüler\tabularnewline
	70--80 & Mittelleicht &\tabularnewline
	80--90 & Leicht &\tabularnewline
	90--100 & Sehr leicht & 11-jährige Schüler\tabularnewline
	\bottomrule
\end{longtable}


\includegraphics[width=\linewidth]{Grafiken/Readability_line.png}\\

Die Hypothese kann nicht bestätigt werden. Eine Aufschlüsselung des
Lesbarkeitsindex nach Politikern könnte in weiter Forschung dennoch Aufschluss über
Veränderungen auf personaler Ebene geben.\\
\newpage

\subsection{Die Inhalte sind seit dem Einzug der AfD einseitiger geworden}

Hypothese kann verworfen werden. Die Themen sind sogar eher gleicher verteilt. 

\begin{figure} [h]
	\subfigure[2009]{\includegraphics[width=0.29\textwidth]{Grafiken/Inhalt09.png}}
	\subfigure[2013]{\includegraphics[width=0.29\textwidth]{Grafiken/Inhalt13.png}}
	\subfigure[2017]{\includegraphics[width=0.29\textwidth]{Grafiken/Inhalt14.png}}
	\caption{Vergleich der Inhalte}
\end{figure}

\subsection{Inhaltliche Distanz zum TOP}

Die inhaltliche Distanz zum Tagesordnungspunkt wird durch einfaches Auszählen gemessen. Jeder Tagesordnungspunkt der Stichprobe wurde kodiert mit einem oder mehreren Inhaltsvariablen. Die Distanz ist für uns definiert durch die Häufigkeit von Redeinhalten, die vom Inhalt des Tagesordnungspunktes abweichen, geteilt durch die Anzahl aller Reden. \\
 
\includegraphics[width=\linewidth]{Grafiken/distop.png}\\


\subsection{Sentiment}

Die diktionär-basierte Sentimentanalyse hat keine nutzbaren Ergebnisse geliefert. Verwendet wurde das Wörterbuch der Uni- Leipzig (Quelle:), das gewichtete positive und negative Wörter enthält. \\

\includegraphics[width=\linewidth]{Grafiken/Sentimentscore.png}\\
\\

\subsection{Reliabilität der Stichprobe} 
\begin{table}[ht]
	\centering
	\begin{tabular}{rrrrr}
		\hline
		Variable & Mehmet und Nils & Mehmet und Vivian & Vivian und Nils & Durchschnitt \\ 
		\hline \hline
		v105 & / & 0.94 & / & / \\ 
		v300 & 0.93 & 0.75 & 1.00 & 0.83 \\ 
		v219 & / & / & 0.93 & / \\ 
		v206 & 0.93 & 0.94 & 1.00 & 0.96 \\ 
		v205 & 0.79 & 0.88 & 0.93 & 0.89 \\ 
		v204 & / & / & 1.00 & / \\ 
		v203 & 0.93 & 0.94 & 0.71 & 0.86 \\ 
		v202 & 0.79 & 0.75 & 0.93 & 0.81 \\ 
		v201 & 0.86 & 0.81 & 0.93 & 0.85 \\ 
		v119 & 0.93 & 0.94 & 0.86 & 0.91 \\ 
		v116 & 0.86 & 1.00 & 0.79 & 0.93 \\ 
		v115 & 1.00 & 1.00 & 1.00 & 1.00 \\ 
		v114 & 0.86 & 0.81 & 0.86 & 0.83 \\ 
		v113 & 0.93 & 0.81 & 0.86 & 0.83 \\ 
		v112 & 1.00 & 0.94 & 0.79 & 0.89 \\ 
		v111 & 1.00 & 0.88 & 0.79 & 0.85 \\ 
		v110 & 0.86 & 0.88 & 0.93 & 0.89 \\ 
		v109 & 0.86 & 0.88 & 1.00 & 0.92 \\ 
		v108 & 1.00 & 1.00 & 1.00 & 1.00 \\ 
		v107 & 0.79 & 0.69 & 0.86 & 0.74 \\ 
		v106 & 0.93 & 1.00 & 1.00 & 1.00 \\ 
		v104 & 0.93 & 0.94 & 1.00 & 0.96 \\ 
		v103 & 0.79 & 0.88 & 0.86 & 0.87 \\ 
		v102 & 0.79 & 0.62 & 0.71 & 0.65 \\ 
		v101 & 0.57 & 0.50 & 0.71 & 0.57 \\ 
		\hline
	\end{tabular}
\end{table}




